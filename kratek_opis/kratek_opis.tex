\documentclass[a4paper, 11pt]{article}
\usepackage[slovene]{babel}
\usepackage[utf8]{inputenc}
\usepackage[T1]{fontenc}
\usepackage{amsfonts,amsmath,amssymb}
\usepackage{amsthm}
\usepackage{amsmath}
\usepackage{amssymb}

\newtheorem{Izrek}{Izrek}

\begin{document}

\begin{titlepage}
    \begin{center}
        \LARGE
        UNIVERZA V LJUBLJANI\\
        FAKULTETA ZA MATEMATIKO IN FIZIKO\\
        FINANČNA MATEMATIKA\\

        \vspace*{5cm}
        
        Finančni praktikum\\
        \huge
        \textbf{The power of two or more choices}

        \vspace*{6cm}

        \Large
        Avtorja:\\
        Nejc Kumer in Neža Lesnjak
    \end{center}
\end{titlepage}
    
\section{Navodila}

Imamo $n$ žog in $n$ košev. Žoge polagamo v koše na spodnje načine:

\begin{itemize}
    \item za vsako žogo izberemo naključen koš in žogo vržemo vanj.
    \item za vsako žogo naključno izberemo dva izmed košev in žogo vržemo v tistega, 
    ki smo ga do tega trenutka manjkrat zadeli.
    \item za vsako žogo naključno izberemo tri izmed košev in žogo vržemo v tistega, 
    ki smo ga do tega trenutka manjkrat zadeli.
    \item ...
\end{itemize}

Zanima nas maksimalna zasedenost, torej število zadetkov v tisti koš, ki smo ga največkrat zadeli.
Eksperimentalno analizirajte te naključne postopke.\\
Kaj se zgodi, če imamo $n$ košev in $2n$, $3n$, $4n$ ... žog?
V tem primeru nas zanima tudi minimalna zasedenost.
Za $n$ izbiramo velike vrednosti.

\section{Opis problema}

Problem $n$ žog in $n$ košev je v teoriji verjetnosti znan problem, ki ima veliko aplikacij tudi v drugih vedah, predvsem v računalništvu.\\
Delitev žog v koše je lahko popolnoma naključna ali delno naključna.
V zgoraj naštetih primerih delitve gre le v prvi točki za popolnoma naključno delitev.
V vseh naslednjih točkah imamo delno naključno delitev - naključno izberemo le določeno število košev in nato žogo položimo v tistega izmed izbranih,
ki smo ga do tega trenutka najmanjkrat zadeli. To v angleškem jeziku poimenujemo \emph{the power of
two or more choices} - odvisno od tega, koliko košev naključno izberemo.

\section{Pričakovani rezultati}

Če imamo torej $n$ košev in $n$ žog v prvem delu naloge, ko za vsako žogo naključno izberemo koš in vržemo žogo vanj,
za koš z največ žogami pričakujemo rezultat iz spodnjega izreka.

\begin{Izrek}
    Koš z največ žogami ima $\Omega (\frac{\log{n}}{\log{\log{n}}})$ žog z verjetnostjo $1 - \frac{1}{\sqrt[3]{n}}$.
\end{Izrek}

V drugem delu naloge, torej v primeru, ko naključno izberemo $d$ košev (in $d \geq 2$), ter žogo položimo v bolj praznega od njih,
pa bomo pričakovali spodnji rezultat.

\begin{Izrek}
    Koš z največ žogami ima $\frac{\log {\log{n}}}{\log{d}} + O(1)$ žog z verjetnostjo blizu 1.
\end{Izrek}

Rezulata iz slednjega izreka je skoraj eksponentno manjši kot rezultat iz prvega izreka za popolnoma naključno delitev žog v koše.

V nalogi bova preverila, kako se rezultati maksimalne zasedenosti spreminjajo, če žoge razporejamo na različne, zgoraj omenjene načine
in kako, če imamo $n$ košev in $n$, $2n$, $3n$, ... žog.

\section{Načrt za nadaljnje delo}
V nadaljevanju bova za reševanje problema uporabila programski jezik \emph{Python} ali \emph{R}.
Pripravila bova program in ga z eksperimentiranjem preizkusila na konkretnih podatkih. Zanimalo naju bo,
koliko bodo rezultati eksperimentov odstopali od predvidenih rezultatov glede na zgornja izreka.


\end{document}